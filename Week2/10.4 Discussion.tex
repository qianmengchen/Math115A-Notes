\documentclass[11pt]{article}
\usepackage{amsmath}
\usepackage{setspace}
\usepackage{amsfonts}
\usepackage{amssymb}

\iffalse
\[
\begin{bmatrix}
\end{bmatrix}
\]
\fi

\begin{document}
\doublespacing
New subspaces from old:\\
If $V$ a vector space, $u,w$ are subspaces, $u\cap w$ is also a subspace. \\
We even showed this for any number of subspaces:\\
If $w_i, i\in I$ are all subspaces, the $\bigcap\limits_{i\in I}w_i$ is a subspace.\\
Here $I$ is called an indexing set.\\

One next homework, we look at $u\cup w$\\
Hint: Try $V=\mathbb{R}^2$\\
$u=$x-axis$=$span$((1,0))=\{(x,0)|x\in \mathbb{R}\}$
$w=$y-axis\\
sums of subspaces:\\
Have $V$ a vector space, $u, w$ subspaces. \\
Then $u+w=\{u+w|u\in U, w\in W\}=\{v\in V|$v can be written as $v=u+w, u\in U, w\in W\}$\\
Let's prove that $u+w$ is a subspace, need to show:
\begin{enumerate}
	\item $0_v\in u+w$
	\item If $v_1, v_2\in u+w$, then $v_1+v_2\in u+w$
	\item If $c\in F$ and $v\in u+w$ then $c\cdot v\in u+w$
\end{enumerate}
\begin{itemize}
	\item $0_v\in u+w$ because $0_v = {0_v}_{\, in\, u}+{0_v}_{\, in\, w}$
	\item Say $v_1, v_2\in u+w$, want $v_1+v_2\in u+w$\\
	That means $v_1=u_1+w_1$ for some $u_1\in u, w_1\in w$\\
	$v_2=u_2+w_2$ for some $u_2\in u, w_2\in w$\\
	So $v_1+v_2=(u_1+w_1)_(u_2+w_2)=(u_1+u_2)_{\, in\, u}+(w_1+w_2)_{\, in\, w}$\\
	\item closed under scalar mult is an exercise
\end{itemize}
Fact: $U+W$ is the smallest subspace of $V$ that contains both $u$ and $w$. More formally: $u\in u+w$, $w\in u+w$\\
If $X\in V$ is a subspace and $U\in X, W\in X$ then $U+W\in X$\\
pf: we won't do the whole thing, will show $U\subset U+W$\\
Let's show $U\subset U+W$\\
i.e. if $x\in U$, then $x\in U+W$ ==> $x=u+w$\\
$x=x+0$, $x$ in $U$ and 0 in $W$\\
Note: saying $v\in U+W$ means $v=u+w$ for some $u\in U, w\in W$\\
Fact: if $U\cap W=\{0\}$\\
then if $v\in U+W$, there's exactly one choice of $u, w$ to make $v=u+w$\\
pf: suppose $v+u_1+w_1$ and $v=u_2+w_2$, we'll show $u_1=u_2$, $w_1=w_2$\\
So $u_1+w_2=u_2+w_2$\\
$u_1-u_2=w_1-w_2$\\
So $u_1-u_2\in U\cap W$, so $u_1-u_2=0$ so $u_1=u_2$. similarly, $w_1=w_2$\\
In this case, your book says $U+W$ is a direct sum ("direct" means $U\cap W=\{0\}$)\\
Take $V, W$ any two vector spaces (over the same field F), define a new vector space $V\oplus W$ called the direct sum of $V, W$\\
vectors in $V\oplus W$ are ordered pairs $(v, w)$\\
$(x, y)+(a, b)=(x+a, y+b)$\\
$c\cdot (x, y)=(cx, cy)$\\
For $U+W$, $U, W$ were both subspaces of $V$\\
If $v\in U+W$, might be many ways to write $v+u+w$\\
For $U\oplus W$, $u, w$ don't need to be related ($\mathbb{R}\oplus [-1, 1]$ is ok), $(x, y)=(w, z) \leftrightarrow x=w, y=z$\\
Fact: If $U, W$ are subspaces of $V$, and $U\cap W=\{0\}$\\
Then there's an invertible linear map. $T: U\oplus W\rightarrow U+W$ (we call these isomorphic vector spaces)\\
We always have a linear map, $T: U\oplus W\rightarrow U+W$ defined by $T(u, w)=u+w$. (True even if $U\cap W\neq \{0\}$). This map is always surjective.\\
Need to show:\\
For any $y\in U+W$, $\exists x\in U\oplus W$ s.t. $T(x)=y$\\
since $y\in U+W$, then $y=U+W$ for some $u\in U, w\in W$\\
So $y=T(u, w)$ (the $x$ above is $(u, w)$)\\
To show injective, we'll use that $U\cap W=\{0\}$\\
Need: If $T(u_1, w_1)=T(u_2, w_2)$, then $(u_1, w_1)=(u_2, w_2)$\\
If $T(u_1, w_1)=T(u_2, w_2)$, that means $u_1+w_1=u_2+w_2$\\
we just showed this means $u_1=u_2, w_1=w_2$
\end{document}
