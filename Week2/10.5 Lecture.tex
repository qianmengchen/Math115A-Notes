\documentclass[11pt]{article}
\usepackage{setspace}
\usepackage{amsmath,amsthm,amssymb,amsfonts}

\DeclareMathOperator{\Span}{span}

\iffalse
\[
\begin{bmatrix}
\end{bmatrix}
\]
\fi

\begin{document}
\doublespacing
\section*{Coordinate Theorem}
let $V$ be a VS/F with basis $\mathbb{B}=\{v_1\ldots v_n\}$, and $v\in V$. \\
Then $\exists \,! \,{\alpha}_1\ldots {\alpha}_n\in F$ s.t. $v={\alpha}_1v_1+\ldots {\alpha}_nv_n$. \\
($\exists\,!\rightarrow$exists unique)\\
\begin{proof}
$\Span\{\mathbb{B}\}=V$;\\
$\mathbb{B}$ satisfies $\Span\{\mathbb{B} \}=V$, $u, v\in $Span($\mathbb{B}$)\\
$\alpha_1\ldots\alpha_n$ unique\\
Suppose 
${\alpha}_1v_1\ldots {\alpha}_nv_n=v=\beta_1v_1+\ldots+\beta_nv_n$, $\,\beta_1\ldots\beta_n\in F$\\
So, $({\alpha}_1-\beta_i)v_1+\ldots+({\alpha}_n-\beta_n)v_n=0$\\
Hence ${\alpha}_i=\beta_i$, $\forall i$ as $\mathbb{B}$ is linearly independent.\\
\end{proof}

Question: is theorem true if $V$ is a VS/F with basis $\mathbb{B}$?\\
Important Exercise: let $V$ be a VS/F, $v_1,\ldots, v_n\in V$. \\
Then $\Span\{v_1\ldots v_n\}$=$\Span\{v_2\ldots v_n\}$ iff $v_1\in \Span\{v_2\ldots v_n\}$\\

\section*{Induction}
To prove a statement by induction, let P(n) be statement for any $n\in \mathbb{Z}^+$ that is either true of false for each n\\
To show P(n) is true $\forall n$\\
To do that you first prove P(1) is true\\
Now suppose P(n) is true (the induction hypothesis)\\
Then, show P(n) $\rightarrow$ P(n+1)\\
{[}Note: $n$ must be arbitrary{]}\\
{[}Note: can stand at any $n$ e.g. $n=0$ {]}]\\
Toss Out: let $V$ be a VS/F and suppose that $V$ can be spanned by a finite number of vectors, then $V$ is a fdvs/F\\
More precisely, if $v_1\ldots v_n\in V$ s.t. $\{v_1\ldots v_n\}$ spans $V$, then a subset of $\{v_1\ldots v_n\}$ is a basis for $V$\\

\begin{proof}
	If $V=0$, done by definition\\
So we may assume $V\neq 0$ and $V=\Span\{v_1\ldots v_n\}$\\
We prove the theorem by induction on $n$\\
$n=1$ as $V\neq 0, v_1\neq 0$\\
Hence $\{v_1\}$ is a basis for $V$ since ${\alpha} v_1=0\rightarrow {\alpha}=0$\\
hence $n=1$ is true\\
Induction hypothesis: suppose span($w_1\ldots w_n$)=$v$ then a subset of $\{w_1\ldots w_n\}$ is a basis\\
To show the result holds if $V=\Span\{v_1\ldots v_{n+1}\}$\\
if $\{v_1\ldots v_{n+1}\}$ is linearly independent, then a basis by definition and done\\
Suppose $\{v_1\ldots v_{n+1}\}$ is linearly dependent\\
Then $\exists \alpha_1\ldots\alpha_{n+1}\in F$ not all zero\\
s.t. $\alpha_1v_1+\ldots+\alpha_{n+1}v_{n+1}=0$\\
Changing notation, we may assume that $\alpha_{n+1}\neq 0$ so ${\alpha_{n+1}}^{-1}$ exists
Then $v_{n+1}=-{\alpha_{n+1}}^{-1}\alpha_1v_1-\ldots\-{\alpha_{n+1}}^{-1}\alpha_nv_n\in \Span\{v_1\ldots v_n\}$\\
By important exercise\\
$V=\Span\{v_1\ldots v_{n+1}\}=\Span\{v_1\ldots v_n\}$\\
By the case $n$, $\{v_1\ldots v_n\}$. Hence $\{v_1\ldots v_{n+1}\}$ contains a basis for $V$.
\end{proof}

\section*{Example}
\begin{enumerate}
	\item let $e_i=(0, \ldots, 0, 1, 0, \ldots, 0)\in F^n$ \& $S_n:=\{e_1\ldots e_n\}\subset F^n$\\
	let $V\in F^n$ then $\exists \alpha_1\ldots \alpha_n\in F$ s.t. $v=(\alpha_1\ldots \alpha_n)=\alpha_1e_1\ldots \alpha_ne_n$\\
	So $V=\Span\{S\}$ if $0=\alpha_1e_1+\ldots+\alpha_ne_n=(\alpha_1\ldots\alpha_n)=(0,\ldots, 0)$\\
	$\rightrightarrows \alpha_i=0\, \forall\, i$\\
	$\therefore$ $S_n$ is a basis called Standard Basis for $F^n$\\
	More generally, let $e_{ij}\in F^{m\times n}$ be the matrix with 0's in all entries, except the ij\textsuperscript{th} place where it is 1.\\
	Then $S=S_{m, n}=\{e_{ij}|1\leq i \leq m, \, 1 \leq j \leq n\}$ is a basis for $F^{m\times n}$\\
	Called standard basis for $F^{m\times n}$
	\item let $V=F[t]$ if $f\in V$, then $\exists \alpha_0\ldots\alpha_n\in F$ s.t. $f=\alpha_0+\alpha_1t+\ldots+\alpha_nt^n$\\
	So, $\Span\{1, t, t^2, \ldots\}=F[t]$ \& $\{1, t, t^2, \ldots\}$ is linearly independent hence a basis for $F[t]$\\
	$\alpha_0+\alpha_1t+\ldots+\alpha_nt^n=0$, then $\alpha_i=0$ $\forall i$
	\item $F[t]_n=\{f\in F[i]\,|\,$degree $f\leq n$ or $f=0\}\subset F[t]$\\
	$F[t]_n=\Span\{1, t, \ldots, t^n\}$ \& $\{1, t, \ldots, t^n\}$ is linearly independent on $\{1, t, t^2, \ldots\}$
	\item $V=\mathbb{C}$ as a VS/R then $\{1, \sqrt{-1}\}$ is a basis for $\mathbb{C}$ as a VS/R\\
	$\alpha 1+\beta\sqrt{-1}$, $\alpha, \beta\in \mathbb{R}$ uniquely\\
	$V=\mathbb{C}$ as a VS/$\mathbb{C}$ then $\{1\}$ is a basis of $\mathbb{C}$ as a VS/$\mathbb{C}$\\
	More generally, $F$ is a field $F$ is a VS/F with basis $\{V\}$ if $V\neq 0$\\
	if $\alpha\in F$ then $\alpha=\alpha v^{-1}v$ as $v$ has a mult inverse\\
	e.g. $\{\pi\}$ is a basis for $\mathbb{R}$ as a VS/R
	\item $\{e^{-x}, e^{3x}\}$ is a basis for $V=\{f\in \mathbb{C}^2\, |\, f''-2f'-3f=0\}$
	\item Given $v_1, \ldots, v_m \in F^n $, know how to find $\Span\{v_1, \ldots, v_m\}$ and a basis for the span
\end{enumerate}

\section{Replacement Theorem}
Let $V$ be a VS/F \& $\{v_1\ldots v_n\}$ is a basis for $V$\\
Suppose $v\in V$ and $v=\alpha_1v_1+\ldots+\alpha_nv_n$, $\alpha_1\ldots\alpha_n\in F$ with $\alpha_i\neq 0$, then $\{v_1, \ldots, v_{i-1}, v_i, v_{i+1}, \ldots, v_n\}$ is a basis for $V$
\end{document}
